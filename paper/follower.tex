
\subsection{Follower Strategic Responses}\label{app:follower-robustness}
We study two types of follower strategic responses.
First, we show that pro-rata haircuts can lead to low leverage users responding by leaving an exchange earlier than higher leverage users.
This imposes a negative feedback loop as the exchange's remaining users are higher risk.
Secondly, we study traders who aim to add liquidity to the exchange to cover a deficit.
These traders are speculating on profits that can be made after an ADL event.
We show that such users are incentivized to wait long than the exchange solvency time.

\subsubsection{Adverse Selection Under Pro-Rata}\label{app:follower-adverse}
We model the ADL interaction as a repeated Stackelberg game: in every round $t$ the exchange (leader) moves first and the surviving winners (followers) best respond.
Each stage looks like a one-round Stackelberg problem, but the outcomes feed into subsequent rounds through the evolving of winner equities $e_{T,i}$ and the winning set $\mathcal{W}_t$.
At the start of round $t$ the exchange publicly commits to a severity/haircut rule (\eg~queue, pro-rata, RAP) that maps the realized deficit $D_t$ and winner book $\mathcal{W}_t$ to haircut shares.
After observing $\theta_t$ and anticipating $D_t$, each winner chooses whether to keep its position active (accepting the induced haircut) or to exit/migrate, receiving outside option $u_0$.
Payoffs are $U_i^{(\pi)}=\mu_i-\Expect[H_{t,i}^{(\pi)}]$; type $i$ exits whenever this falls below $u_0$.

A \emph{death spiral} occurs when pro-rata haircuts force the safest (low-leverage) winners to churn first, shrinking $W_t$, which in turn raises the future haircut share for the remaining winners, triggering additional exits and further eroding liquidity.
We show that RAP breaks this feedback loop by tilting the follower game against high-leverage accounts.

\iparagraph{Setup.}
Let $i$ index a profitable trader with effective equity $e_{t,i}$, leverage $\lambda_{t,i}$, and expected per-round utility $\mu_i$.
Normalize severity and deficits by $\bar\theta=\Expect[\theta_t]$, $\bar D=\Expect[D_t]$, and write $\bar W=\Expect[W_t]$ for the expected equity mass of winners.
Under pro-rata, the haircut share of $i$ is $s^{\mathrm{PR}}_{t,i}=e_{t,i}/W_t$, so the realized haircut mass is
\[
H^{\mathrm{PR}}_{t,i}=\theta_t D_t\, s^{\mathrm{PR}}_{t,i}.
\]
For RAP with a nondecreasing weight $g$, the share becomes
\[
s^{\mathrm{RAP}}_{t,i}=\frac{e_{t,i}\,\lambda_{t,i} g(\lambda_{t,i})}{\sum_{j\in W_t} e_{t,j}\,\lambda_{t,j} g(\lambda_{t,j})},
\qquad
H^{\mathrm{RAP}}_{t,i}=\theta_t D_t\, s^{\mathrm{RAP}}_{t,i}.
\]
\iparagraph{Risk-intensity comparator.}
Define the equity‑weighted market risk intensity at round $t$ by
\[
\mu^{(g)}_t\ :=\ \frac{\sum_{j\in W_t} e_{t,j}\,\lambda_{t,j} g(\lambda_{t,j})}{\sum_{j\in W_t} e_{t,j}}.
\]
\begin{lemma}[When RAP burdens a trader less than Pro‑Rata]\label{lem:rap-vs-pr-share}
For any winner $i$,
\[
s^{\mathrm{RAP}}_{t,i}\ \le\ s^{\mathrm{PR}}_{t,i}
\quad\Longleftrightarrow\quad
\lambda_{t,i}\,g(\lambda_{t,i})\ \le\ \mu^{(g)}_t.
\]
If $g$ is strictly increasing and $\lambda_{t,i}$ is strictly below the equity‑weighted market average in the $g$‑scale, the inequality is strict.
\end{lemma}
\begin{proof}
Compute
\[
\frac{s^{\mathrm{RAP}}_{t,i}}{s^{\mathrm{PR}}_{t,i}}
\ =\
\frac{e_{t,i}\lambda_{t,i} g(\lambda_{t,i})/\sum_j e_{t,j}\lambda_{t,j} g(\lambda_{t,j})}{e_{t,i}/\sum_j e_{t,j}}
\ =\
\frac{\lambda_{t,i} g(\lambda_{t,i})}{\mu^{(g)}_t}.
\]
The claim follows immediately.
\end{proof}

\iparagraph{Participation thresholds.}
Each trader type $i$ has a reservation utility $u_0$: the expected per-round payoff it can secure outside the ADL venue (\eg~by migrating flow to another exchange, posting liquidity in a different product, or simply investing idle cash in a risk-free instrument).
We treat $u_0>0$ as exogenous and, unless stated otherwise, common across trader types.
We note that heterogeneity in reservation utility can be captured by indexing it as $u_{0,i}$ without changing the argument.

A trader remains active only if the ADL-adjusted payoff exceeds this fallback value.
We formalize this by defining the net utility under policy $\pi$ as
\[
U_i^{(\pi)} := \mu_i - \Expect[H_{t,i}^{(\pi)}],
\]
and saying that $i$ participates in a given regime iff $U_i^{(\pi)} \ge u_0$.
Equivalently, the \emph{participation threshold} is the maximum haircut burden $\Expect[H_{t,i}^{(\pi)}]$ that keeps $i$ indifferent, namely $\mu_i-u_0$.
\begin{corollary}[Pro‑Rata death spiral vs.\ RAP retention]\label{cor:death-spiral}
Fix a type $i$ and suppose there is a set of rounds of positive probability on which $\lambda_{t,i} g(\lambda_{t,i})\le \mu^{(g)}_t$ (so $s^{\mathrm{RAP}}_{t,i}\le s^{\mathrm{PR}}_{t,i}$ by Lemma~\ref{lem:rap-vs-pr-share}).
If, over the same distribution of rounds,
\[
\mu_i - \Expect\big[\theta_t D_t\, s^{\mathrm{PR}}_{t,i}\big]\ <\ u_0
\ \ \le\ \
\mu_i - \Expect\big[\theta_t D_t\, s^{\mathrm{RAP}}_{t,i}\big],
\]
then the participation constraint fails under pro‑rata but holds under RAP: type $i$ exits in the pro‑rata regime while remaining under RAP.
\end{corollary}

\iparagraph{Examples and calibration.}
Consider two winners, $L$ (low leverage) and $H$ (high leverage), who trade for two rounds.
Equities are $(e_L,e_H)=(60,40)$, leverage levels are $(\lambda_L,\lambda_H)=(2,6)$, expected per-round utilities are $(\mu_L,\mu_H)=(12,40)$, and the outside option is $u_0=2$.
Each round the deficit equals the total haircut budget ($\theta_t=1$) with $D_1=40$ and $D_2=30$.

\emph{Pro-rata.}
Shares equal equity weights: $s_i^{\mathrm{PR}}=e_i/(e_L+e_H)$, so $s^{\mathrm{PR}}=(0.6,0.4)$ even though both traders lose the \emph{same haircut factor} $h^{\mathrm{PR}}=H_i^{\mathrm{PR}}/e_i=\theta_t D_t/W_t=0.4$.
Round~1 haircut masses are $H^{\mathrm{PR}}=(24,16)$ and utilities are $U^{\mathrm{PR}}=(\!-12,24)$.
Trader $L$ churns because $U_L^{\mathrm{PR}}<u_0$, leaving only $H$ for round~2.
With $W_2=40$ the next deficit $D_2=30$ forces $H_{2,H}^{\mathrm{PR}}=30$ (i.e., haircut factor $0.75$), giving $U_{2,H}^{\mathrm{PR}}=10$; solvency is preserved but the equity base has already halved, so any larger $D_2$ would wipe out the last winner, illustrating the death spiral.

\emph{RAP with $g(\lambda)=\lambda$.}
Weights scale as $e_i\lambda_i^2$, so $L$'s share collapses to $s_L^{\mathrm{RAP}}=240/(240+1440)\approx0.14$ and $H$'s share rises to $0.86$.
The round~1 haircut factors become $h_L^{\mathrm{RAP}}\approx0.095$ and $h_H^{\mathrm{RAP}}\approx0.86$ (masses $H^{\mathrm{RAP}}=(5.7,34.3)$), yielding $U^{\mathrm{RAP}}=(6.3,5.7)>u_0$, so both types remain for round~2.
With both accounts active in round~2, shares remain tilted toward $H$ and haircuts $(4.3,25.7)$ keep both traders above $u_0$, preventing churn and stabilizing $W_t$.
This concrete two-player, two-round example mirrors the equilibria predicted by Appendix~\ref{app:mh-example} and the empirical replay in \S\ref{sec:numerics}: pro-rata drives the safest capital away first, whereas RAP reallocates the burden toward high-leverage accounts and keeps the equity base intact.
\subsubsection{Waiting Game and MDIC-NW}\label{app:follower-wait}

\iparagraph{Game.}
When an ADL event creates a deficit $D_t$, the exchange moves first: it announces the contemporaneous severity $\theta_t$ (hence the haircut budget $\theta_t D_t$) together with the liquidity premium $\kappa_t$ it is willing to pay per unit of external capital.\footnote{In practice, this occurs via either an increase in the expected payment for users who stake to an insurance fund (\eg~\cite{DriftInsuranceFund}) or add assets to an HLP/LLP style vault~\cite{HyperliquidHLPVaults,LighterWhitepaper}}
A Backstop Liquidity Provider (BLP) then decides whether to intervene immediately by injecting any $q_t\in[0,D_t]$ units, or to wait $u\ge 1$ additional rounds before posting the same quantity.
Waiting exposes the BLP to time discounting (with a factor $\beta^u$) and to the future premium schedule $(\theta_{t+u},\kappa_{t+u})$.
The BLP’s payoff from intervening in round $t+u$ with size $q$ is $\beta^u q(\theta_{t+u}-\kappa_{t+u})$, while the exchange’s objective is to restore solvency before a deadline by ensuring that the entire deficit is filled (so delay harms solvency).

This Stackelberg game involves the exchange posting incentives as leader and BLPs optimally choose a stopping time and quantity as a follower.
The setup captures the ``waiting game’’ intuition: unless the contract guarantees non-negative per-unit surplus right now, rational liquidity providers will defer, pushing resolution beyond the solvency window.
We note that this resembles other waiting games in Maximal Extractable Value (MEV) auctions that have been studied empirically on live systems~\cite{messias2025express,mamageishvili2025timeboost}.

\iparagraph{Per‑unit surplus and waiting.}
Let a Backstop Liquidity Provider (BLP) be able to absorb up to $D_t$ units at time $t$.
Write the per‑unit liquidity premium as $\kappa_t\ge 0$ (execution cost plus risk per unit, e.g., $\kappa_t=\Gamma_t/D_t$ when the premium is linear in size) and define the per‑unit net surplus
\[
\delta_t\ :=\ \theta_t - \kappa_t.
\]
If the BLP executes $q\in[0,D_t]$ units at time $t$, the immediate surplus is $q\,\delta_t$; deferring the same $q$ to a later time $t+u$ yields discounted surplus $\beta^u q\,\delta_{t+u}$.

\iparagraph{No-wait condition.}
We say the exchange enforces a per‑unit ``No‑Wait'' constraint when $\delta_t\ge 0$ (equivalently, $\theta_t D_t \ge \Gamma_t$ in the linear‑premium case).

\begin{lemma}[Per‑unit No‑Wait implies immediate action]\label{lem:no-wait}
Suppose $\beta\in(0,1]$, the exchange enforces $\delta_t\ge 0$, and the net per‑unit surplus is nonincreasing over time: $\delta_{t+u}\le \delta_t$ for all $u\ge 0$ (e.g., when $\theta_{t+u}\le \theta_t$ and $\kappa_{t+u}\ge \kappa_t$).
Then for any $q\in[0,D_t]$, executing $q$ immediately at $t$ weakly dominates waiting:
\[
q\,\delta_t\ \ge\ \beta^u\,q\,\delta_{t+u}\qquad\forall\,u\ge 0.
\]
In particular, the BLP’s optimal stopping time is $\tau^\star=t$ and, if capacity allows, $q_t^\star=D_t$.
\end{lemma}
\begin{proof}
By assumption, $\delta_{t+u}\le \delta_t$ and $\beta^u\le 1$ for all $u\ge 0$, hence $\beta^u q\,\delta_{t+u}\le q\,\delta_t$ for any fixed $q\in[0,D_t]$.
Summing over an optimal decomposition of $D_t$ into infinitesimal units yields the stated dominance and the immediate‑execution optimality.
\end{proof}

\begin{corollary}[No‑Wait bounds solvency recovery]\label{cor:nowait-solv}
Let $\tau_{\mathrm{def}}$ denote the default time and recall the solvency recovery clock $\tau_{\mathrm{solv}}$ from \S\ref{subsec:opposite-orders}.
If the exchange enforces $\delta_t\ge 0$ for all rounds between $\tau_{\mathrm{def}}$ and the first round in which the deficit is zero, then $\tau_{\mathrm{solv}}=\tau_{\mathrm{def}}+1$.
Hence solvency is restored within a single round of the default event.
\end{corollary}
\begin{proof}
Lemma~\ref{lem:no-wait} implies that at $\tau_{\mathrm{def}}$ the BLP injects $q_{\tau_{\mathrm{def}}}^\star=D_{\tau_{\mathrm{def}}}$, so the entire deficit is covered immediately.
Therefore the insurance fund (or deficit buffer) reaches the safety level $\delta$ after the same round, yielding $\tau_{\mathrm{solv}}=\tau_{\mathrm{def}}+1$.
\end{proof}

\noindent This result states that if the no-wait condition is enforced, BLPs will inject liquidity such that the solvency time is minimized.
In theory, an exchange can use some of its future revenue or profits (\eg~via token issuance) to enforce the no-wait constraint.
\iparagraph{Proof of Proposition~\ref{prop:no-wait}}
\begin{proof}
Let $U(\tau, q) = \beta^{\tau-t} q (\theta_{\tau} - \kappa_{\tau})$ be the discounted utility of a BLP who provides liquidity $q$ at time $\tau \ge t$.
Here, $\theta_{\tau}$ represents the payment per unit of liquidity (derived from the ADL severity) and $\kappa_{\tau}$ represents the cost per unit (liquidity premium).
The BLP solves the optimal stopping problem (see, \eg~\cite{Peskir2006}):
\[
\max_{\tau \ge t, q \in [0, D_t]} \Expect_t\left[ U(\tau, q) \right].
\]
This objective captures the trade-off between acting immediately to capture the current spread versus waiting for potentially higher future premiums, discounted by the time cost of delay.
Assuming the BLP is risk-neutral and has capacity $q=D_t$, the condition for immediate stopping at $\tau=t$ is that the immediate payoff exceeds the expected discounted future value:
\[
D_t (\theta_t - \kappa_t) \;\ge\; \beta \Expect_t[ V_{t+1}(D_{t+1}) ],
\]
where $V_{t+1}(D_{t+1}) = \max_{\tau \ge t+1, q \in [0, D_{t+1}]} \Expect_{t+1}[U(\tau, q)]$ is the value function from $t+1$ onwards.
Substituting the per-unit surplus $\delta_t = \theta_t - \kappa_t$ and rearranging gives the condition:
\[
\theta_t D_t \;\ge\; \Gamma_t + \beta \Expect_t[ V_{t+1}(D_{t+1}) ].
\]
In the simplified case where the BLP compares acting now vs. acting at $t+\Delta t$, and assuming capacity constraints are non-binding, the condition reduces to comparing marginal costs.
Specifically, if the cost of waiting is strictly positive (due to discounting $\beta < 1$ or decreasing surplus $\delta_{t+\Delta t} < \delta_t$), then the optimal strategy is to act immediately.
If, however, $\theta_t$ decays rapidly such that $\theta_t D_t > \Expect[\theta_{t+\Delta t} D_{t+\Delta t}]$, but the premium $\Gamma_t$ makes immediate action costly, the inequality flips.
Rearranging the condition in the proposition statement:
\[
\theta_t D_t + \Gamma_t \le \Expect_t[\theta_{t+\Delta t} D_{t+\Delta t}]
\]
implies that the future expected payout (even after discounting) is higher than the current payout net of costs, incentivizing delay.
Thus, to enforce no-wait, the exchange must ensure the reverse inequality holds.
\end{proof}