\section{Moral Hazard Results}\label{app:proofs}
 
\subsection{Queue maximizes top‑winner damage: formal proofs and gaps}\label{app:queue-vs-pr-top}

\iparagraph{Setup.}
Fix a time \(T\).
Let winners' equities be sorted \(e_{(1)}\ge e_{(2)}\ge\cdots\ge e_{(k)}>0\) and denote \(W_T=\sum_{i=1}^k e_{(i)}\).
A budget‑balanced ADL policy chooses haircuts \(h_i\in[0,1]\) with total haircut mass
\[
\sum_{i=1}^k h_i e_{(i)} \;=\; H \;=\; \theta_\pi D_T,
\]
and produces post‑haircut survivors \(z_i=(1-h_i)e_{(i)}\).
Let \(\Delta_T\) be the largest loser shortfall at the shock.

\iparagraph{Queue minimizes the top survivor.}
\begin{lemma}\label{lem:queue-min-top}
For any feasible haircut vector \(h\), the top‑winner survivor satisfies
\[
e^{\mathrm{post}}_{(1)} \;=\; (1-h_1)e_{(1)} \;\ge\; \max\{e_{(1)}-H,\,0\}.
\]
Equality is attained by the queue rule:
if \(H\le e_{(1)}\), set \(h_1=H/e_{(1)}\) and \(h_i=0\) for \(i>1\);
if \(H>e_{(1)}\), set \(h_1=1\) and continue greedily on \(i=2,3,\dots\).
\end{lemma}
\begin{proof}
Since \(h_i\ge 0\), we have \(h_1 e_{(1)} \le \sum_{i=1}^k h_i e_{(i)} = H\).
Thus \((1-h_1)e_{(1)} \ge e_{(1)}-H\), and the nonnegativity of survivors gives the max with \(0\).
The queue construction achieves this bound by allocating as much of \(H\) as possible to the top account before touching others.
\end{proof}

\iparagraph{Top‑winner PTSR/PMR extremality.}
\begin{proposition}\label{prop:top-ptsr-pmr}
Define the top‑winner ratios
\[
\mathrm{PTSR}^{\mathrm{top}}_T \;=\; \frac{e^{\mathrm{post}}_{(1)}}{D_T},
\qquad
\mathrm{PMR}^{\mathrm{top}}_T \;=\; \frac{e^{\mathrm{post}}_{(1)}}{\Delta_T}.
\]
Among all budget‑balanced haircut vectors with total budget \(H=\theta_\pi D_T\), the queue rule minimizes \(\mathrm{PTSR}^{\mathrm{top}}_T\) and \(\mathrm{PMR}^{\mathrm{top}}_T\).
\end{proposition}
\begin{proof}
Immediate from Lemma~\ref{lem:queue-min-top} by dividing through by the positive constants \(D_T\) and \(\Delta_T\).
\end{proof}

\iparagraph{Gap versus pro‑rata.}
\begin{proposition}\label{prop:gap-pr-vs-queue}
Let \(H\le e_{(1)}\) and consider the pro‑rata rule \(h_i^{\mathrm{PR}}=H/W_T\).
Then
\[
e^{\mathrm{post,PR}}_{(1)} - e^{\mathrm{post,Queue}}_{(1)}
\;=\; H\Big(1-\frac{e_{(1)}}{W_T}\Big)\;\;>\;0\quad\text{whenever}\quad e_{(1)}<W_T.
\]
Consequently,
\[
\mathrm{PTSR}^{\mathrm{top}}_{\mathrm{PR}} - \mathrm{PTSR}^{\mathrm{top}}_{\mathrm{Queue}}
\;=\; \frac{H}{D_T}\Big(1-\frac{e_{(1)}}{W_T}\Big)
\;=\; \theta_\pi\Big(1-\frac{e_{(1)}}{W_T}\Big),
\]
and
\[
\mathrm{PMR}^{\mathrm{top}}_{\mathrm{PR}} - \mathrm{PMR}^{\mathrm{top}}_{\mathrm{Queue}}
\;=\; \frac{H}{\Delta_T}\Big(1-\frac{e_{(1)}}{W_T}\Big).
\]
If \(H>e_{(1)}\), replace \(e^{\mathrm{post,Queue}}_{(1)}\) by \(0\), yielding
\(
e^{\mathrm{post,PR}}_{(1)} - 0 = \big(1-\tfrac{H}{W_T}\big)e_{(1)}
\)
and the corresponding ratio gaps.
\end{proposition}
\begin{proof}
Under pro‑rata, \(e^{\mathrm{post,PR}}_{(1)} = (1-H/W_T)e_{(1)}\).
Under queue with \(H\le e_{(1)}\), \(e^{\mathrm{post,Queue}}_{(1)}=e_{(1)}-H\).
Subtract and simplify to obtain the display; dividing by \(D_T\) (resp.\ \(\Delta_T\)) yields the ratio gaps.
The \(H>e_{(1)}\) case follows since the queue fully exhausts the top account.
\end{proof}

\iparagraph{Discussion.}
Propositions~\ref{prop:top-ptsr-pmr}–\ref{prop:gap-pr-vs-queue} formalize the intuition that queueing concentrates the burden on the most profitable account, while proportional rules preserve the top winner to the greatest extent compatible with budget balance.
The explicit gap scales linearly with the budget \(H\) and with the complement of the top‑winner share \(1-e_{(1)}/W_T\), making the amplification with concentration transparent.

Lemma~\ref{lem:queue-min-top} shows that queue greedily allocates the entire haircut budget to the largest winner, thereby minimizing the post‑ADL survivor \(\omega_T^\pi\) among all feasible allocations when \(H\le e_{(1)}\).
Proposition~\ref{prop:top-ptsr-pmr} converts this extremal survivor property into the PTSR/PMR statements by normalizing with \(D_T\) and \(\Delta_T\), while Proposition~\ref{prop:gap-pr-vs-queue} quantifies the strict gap relative to pro‑rata.
Together these statements exactly recover the inequalities in Proposition~\ref{prop:queue-worst-top-informal}, completing the proof.
\subsection{LLN/EVT regime and scaling proofs}\label{app:lln-evt}
We assume triangular arrays with regularly varying right/left tails (indices $\alpha_+,\alpha_-\in(0,2]$) and slowly varying factors.
The winner/loser counts satisfy $k_n,m_n=\Theta(n)$, and the extreme-value scales coincide with the upper-quantile normalizers \eqref{eq:quantile-scales}: $b_{k_n}^+=F_+^{\leftarrow}(1-1/k_n)$ and $b_{m_n}^-=F_-^{\leftarrow}(1-1/m_n)$.
Under mild mixing, these tails deliver the LLN/EV limits \eqref{eq:lln-scaling}--\eqref{eq:ev-scaling}.
Theorem~\ref{thm:master-ptsr} follows by bounding $\omega_T^\pi$ between the maximum survivor and its capped reverse‑waterfilling counterpart and applying bounded convergence for expectations.

\subsection{Finite‑sample bounds for PTSR}\label{app:finite-sample}
Under Pareto$(\alpha_+)$ winners (or subexponential right tails), quantile bounds and maximal inequalities imply that, with probability $\ge 1-\delta$, $\omega_T\in[c_-(\alpha_+,\delta)\,b_{k_n}^+,c_+(\alpha_+,\delta)\,b_{k_n}^+]$ and $D_T\in[\underline\mu(\delta)\,n,\overline\mu(\delta)\,n]$.
Combining with the reverse‑waterfilling sandwich yields the finite‑sample display in \S\ref{thm:master-ptsr}.
Constants follow from standard EVT quantile approximations (Leadbetter–Lindgren–Rootzén; Resnick; Embrechts–Klüppelberg–Mikosch).

\subsection{Leverage anti‑concentration}\label{app:anti-conc}
Assume $\max_i L_i^-/\ell_n^-\to 0$ and that the loss tail is subexponential/regularly varying.
Then $\Delta_T$ scales as $(\ell_n^-/n)\,b_{m_n}^-$ up to constants.
Counterexamples: a single loser carrying $\Theta(\ell_n^-)$ leverage breaks the scaling.

\begin{theorem}[Master theorem: PTSR scaling]\label{thm:master-ptsr}\label{thm:ev-main}
    Fix $T$ and state $\mathcal P_n$. Let $W_T=\sum(e_T)_+$, $D_T=\sum(-e_T)_+$, and $\omega_T=\max(e_T)_+$. Assume the LLN/EVT regime of \S2--\S3: $W_T/n\xrightarrow{p}\mu_+$, $D_T/n\xrightarrow{p}\mu_-$, and $\omega_T/b_{k_n}^+\xrightarrow{p}c_+$, with $k_n=\Theta(n)$ and $b_{k_n}^+$ the winner--side extreme--value scale.
    Let $\pi$ be any one--round ADL policy satisfying (6)--(7) with severity $H=\theta_n D_T$.
    With high probability,
    \[
    \frac{c_+}{\mu_-}\,\frac{b_{k_n}^+}{\theta_n n}\,(1-o(1))
    \ \ \le\ \ \frac{\omega_T^\pi}{D_T^\pi}\ \ \le\ \ \frac{1}{\mu_-}\,\frac{b_{k_n}^+}{\theta_n n}\,(1+o(1)).
    \]
    Thus $\mathsf{PTSR}_T(\mathcal P_n,\pi) = \Theta(b_{k_n}^+/(\theta_n n))$.
    Finite-sample bounds follow from standard EVT concentration inequalities (e.g., \cite{Embrechts1997}).
    \end{theorem}
    
    \begin{proof}
    Budget balance implies $D_T^\pi=\theta_n D_T = \theta_n \mu_- n(1+o_p(1))$.
    For any rule, $\omega_T^\pi\le \omega_T$. Under standard capped reverse--waterfilling (13)--(14), if $H=O(b_{k_n}^+)\ll W_T=\Theta(n)$, the maximum survivor is preserved: $\omega_T^\pi\ge (1-o_p(1))\omega_T$.
    The ratio scales as $\omega_T/(\theta_n D_T) = \Theta_p(b_{k_n}^+/(\theta_n n))$.
    Bounded convergence extends this to expectations.
    \end{proof}
    
    \begin{corollary}[PTSR decay]\label{cor:A-ptsr}
    Under Theorem~\ref{thm:master-ptsr}, if $\theta_n n/b_{k_n}^+\to\infty$, then $\mathsf{PTSR}\to 0$. Conversely, choosing $\theta_n=C\,b_{k_n}^+/n$ maintains $\mathsf{PTSR}$ at a nonzero constant $c_+/(C\mu_-)$.
    \end{corollary}
    
    \begin{corollary}[PMR scaling]\label{cor:B-pmr}
    Under LLN/EVT and leverage anti--concentration ($\Delta_T \asymp (\ell_n^-/n) b_{m_n}^-$),
    \[
    \mathsf{PMR}_T(\mathcal P_n,\pi)\ =\ \Theta\!\Big(\frac{n}{\ell_n^-}\Big)\cdot \frac{b_{k_n}^+}{b_{m_n}^-}\cdot \frac{1}{\theta_n}.
    \]
    If tails are comparable ($b_{m_n}^-\asymp b_{k_n}^+$), $\mathsf{PMR} \asymp (n/\ell_n^-)/\theta_n$.
    Thus PMR is small when severity is large or loser leverage per trader is high.
    \end{corollary}
    
    \begin{corollary}[Policy ordering]\label{cor:C-policy-order}
    Fix $H=\theta_n D_T \le W_{-1}$. Among fair winner allocators (anonymity, scale invariance, monotonicity), capped reverse--waterfilling maximizes $\omega_T^\pi$ and thus maximizes both PTSR and PMR. Top--first queues minimize $\omega_T^\pi$ and these ratios.
    \end{corollary}
    
    \subsection{Randomized constructions for moral-hazard examples}\label{app:mh-example}
    
    \paragraph{Extreme-value moral-hazard (principal–agent).}
    Fix $\rho\in(0,1)$ and $k_n=\lfloor \rho n\rfloor$. Draw winner equities $Y_i^{(n)}$ i.i.d.\ Pareto$(\alpha_+)$ and loser equities $X_i^{(n)}$ with mean $\mu$.
    Then $M_n^+=\max Y_i^{(n)} \asymp n^{1/\alpha_+}$ while non-max winners sum to $o(M_n^+)$.
    Losers sum to $D_T \approx \mu n$.
    For fixed severity $\theta_n \equiv \bar\theta$, the haircut $H_n \approx \bar\theta \mu n$ exceeds the capacity of non-max winners, forcing the top winner to cover the bulk.
    Post-ADL equity is $(M_n^+ - \bar\theta \mu n)_+ \to 0$ since $n^{1/\alpha_+} \ll n$ for $\alpha_+ < 1$.
    
    \paragraph{Leverage-imbalance construction.}\label{app:mh-leverage}
    Fix leverage masses $\ell_n^- \gg \ell_n^+$. Draw loser equities $X_i^{(n)}$ i.i.d.\ Pareto$(\alpha_-)$, so $D_T \approx M_n^- \asymp n^{1/\alpha_-}$.
    Assign winner leverage $c\,\ell_n^+$ to a random index $I_n$ and distribute the rest evenly.
    Then $\omega_n \asymp (\ell_n^+/\ell_n^-) n$.
    This satisfies Proposition~\ref{prop:excessive-leverage} assumptions, yielding the claimed threshold.
    