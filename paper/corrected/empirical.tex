
\section{Empirical Methodology}\label{app:methodology}
This appendix documents the OSS reproduction pipeline underlying~\S\ref{sec:numerics}.
The goal is to make the empirical objects \emph{file-backed} (every number in~\S\ref{sec:numerics} is computed from on-disk artifacts under \texttt{OSS/out/}) and to keep the two-space hygiene explicit: production ADL executes in contract space, while we measure impacts in wealth space.

\subsection{Data construction}\label{app:methodology-data}

\iparagraph{Data sources.}
We use HyperReplay's ground-truth fill and misc event streams together with clearinghouse snapshots to replay winner account states.
We also use the canonical REALTIME table \texttt{adl\_detailed\_analysis\_REALTIME.csv} to extract winner sets and to form equity/PNL capacity proxies.
The OSS code resolves upstream paths in \texttt{OSS/src/oss\_adl/paths.py} and writes derived artifacts under \texttt{OSS/out/}.

\iparagraph{Wave construction (global time waves).}
We define \emph{global waves} by gap clustering on ADL-fill timestamps:
sort ADL-fill rows by time, start a new wave when the inter-fill gap exceeds \(\texttt{gap\_ms}=5000\)ms, and define wave \(t\) as \([t_{\mathrm{start}}(t),t_{\mathrm{end}}(t)]\).
Global (not per-coin) waves avoid double counting a single solvency episode across multiple markets.

\iparagraph{Loser deficit \(D_t\).}
For each wave, the realized loser deficit is
\[
D_t=\sum_{j \in \mathrm{losers}(t)}(-e_t(j))_+,
\]
computed from loser-side liquidation equity fields.
Concretely, for each \texttt{liquidated\_user} we take the minimum observed \texttt{liquidated\_total\_equity} within the wave and sum the negative minima across users.

\iparagraph{Needed budget \(B_t^{\mathrm{needed}}\).}
For each ADL fill \(k\), we parse the liquidation mark \(p_k^{\mathrm{mark}}\) and execution price \(p_k^{\mathrm{exec}}\) from the raw fill stream (fields \texttt{markPx} and \texttt{px}), together with size \(q_k\), and set
\[
\mathrm{needed}_k:=\lvert p_k^{\mathrm{mark}}-p_k^{\mathrm{exec}}\rvert\cdot \lvert q_k\rvert,
\qquad
B_t^{\mathrm{needed}}:=\sum_{k \in \text{fills in wave }t}\mathrm{needed}_k.
\]
This is an instantaneous bankruptcy-gap proxy: it does not include opportunity cost after the wave ends.

\iparagraph{Two-pass replay and production wealth removed \(H_t^{\mathrm{prod}}(\Delta)\).}
We replay the realized event stream twice for the winner set in each wave:
an ADL-on pass (apply all fills) and a no-ADL pass (skip ADL fills in state updates while keeping the realized price path).
We evaluate equities at \(t_{\mathrm{eval}}=t_{\mathrm{end}}+\Delta\) (after wave end we update only the price path) and compute
\[
H_t^{\mathrm{prod}}(\Delta)=\sum_{u\in W(t)}\bigl(e_{t,\mathrm{end}}^{\mathrm{no\text{-}ADL}}(u;\Delta)-e_{t,\mathrm{end}}^{\mathrm{ADL}}(u;\Delta)\bigr)_+.
\]
The no-ADL counterfactual is not identifiable from public data without an assumption; we use the standard ``hold the realized price path fixed, remove the ADL state update'' counterfactual.

\iparagraph{Production overshoot vs needed.}
We report
\[
O_t(\Delta):=H_t^{\mathrm{prod}}(\Delta)-B_t^{\mathrm{needed}},\qquad
O(\Delta):=\sum_t O_t(\Delta).
\]
A horizon sweep varies \(\Delta\) (scenario parameter) to expose a short-horizon opportunity-cost channel.
The per-horizon totals are written to \texttt{OSS/out/eval\_horizon\_sweep\_gap\_ms=5000.csv} and summarized in \texttt{OSS/out/overshoot\_robustness.json}.

\subsection{Benchmark allocations targeting \(B_t^{\mathrm{needed}}\)}\label{app:benchmarks}
To make ``excessive relative to alternatives'' concrete, we compare production to transparent benchmark allocations that target \(B_t^{\mathrm{needed}}\) wave-by-wave under a profits-only capacity constraint.
For each wave we build a per-user capacity proxy
\[
c_u:=\min\{U(u),E(u)\},
\]
where \(U(u)\) is positive unrealized PNL and \(E(u)\) is positive equity, both taken from the canonical REALTIME table within the wave window.
Benchmarks differ mainly in how they handle contract discreteness:
\begin{itemize}
  \item \emph{Wealth pro-rata (continuous)}: capped pro-rata in wealth-space USD over \(c_u\).
  \item \emph{Vector mirror descent (vector-md)}: a projection-based allocator that produces \(h=x\odot c\) with \(c^\top x=B_t^{\mathrm{needed}}\) and \(x\in[0,1]^n\).
  \item \emph{Contract pro-rata}: standard exchange-style integer allocation proportional to position size.
  \item \emph{Min-max ILP}: MIP solver minimizing maximum haircut percentage.
\end{itemize}
These outputs are written to \texttt{OSS/out/policy\_per\_wave\_metrics.csv} and plotted in Figures~\ref{fig:emp-policy-per-wave}--\ref{fig:emp-cumulative-overshoot}.

\subsection{Diagnostic decomposition (not the \S9 overshoot metric)}\label{app:markout-diagnostic}
This subsection is a diagnostic for interpreting execution timing and for answering common ``markout'' questions; it is not the overshoot metric used in~\S\ref{sec:numerics}.
From the ADLed user's perspective, with signed quantity \(Q\), execution price \(p_{\text{exec}}\), nearest mark at execution time \(p_{\text{mark}}(t)\), and mark at horizon \(p_{\text{mark}}(t+\Delta)\), total markout decomposes as
\[
Q\cdot(p_{\text{mark}}(t+\Delta)-p_{\text{exec}})
=
Q\cdot(p_{\text{mark}}(t)-p_{\text{exec}})
+
Q\cdot(p_{\text{mark}}(t+\Delta)-p_{\text{mark}}(t)).
\]
Because this diagnostic combines executions with time-indexed marks and the public-data replay lacks full clearing/settlement observability, its aggregate need not satisfy a strict two-party zero-sum identity.

\iparagraph{Revenue proxy.}\label{app:revenue-proxy}
To translate haircut allocations into an expected loss of future fees, we use a minimal churn proxy.
For a winner \(u\) in wave \(t\) with USD haircut \(h_{t,u}\) and capacity proxy \(c_u\), define the churn probability
\[
p_{t,u}=1-\exp\!\left(-\beta\,h_{t,u}/c_u\right),
\]
and use this to scale down a simple fee proxy based on notional and per-trade fee rates.
This proxy is intended for within-paper comparisons under a common set of assumptions; it is not an identification claim about venue-level revenue.

