\documentclass[../main_corrected.tex]{subfiles}
\begin{document}

\section{Risk-Aware Pro-Rata (RAP)}\label{app:rap}

\subsection{Examples of Risk-Aware Pro-Rata (RAP) and Next Deficit}\label{app:rap-examples}

In this section, we provide detailed numerical examples illustrating the properties of the RAP rule and the impact of post-haircut shocks.

\iparagraph{RAP weighting example.}
We illustrate the three choices of risk models on an example at $T=2$ where the winners are $\mathcal W_2=\{A,C,E\}$ with effective leverages $\lambda^+_{A,2}\approx 1.031$, $\lambda^+_{C,2}\approx 0.925$, and $\lambda^+_{E,2}\approx 1.548$.
Recall that under pro-rata, the normalized shares are $s^{\mathrm{PR}}_i \propto e_{T,i}$, yielding $(s^{\mathrm{PR}}_A, s^{\mathrm{PR}}_C, s^{\mathrm{PR}}_E) \approx (0.163, 0.728, 0.109)$.
For RAP with $w_i=\lambda_i g(\lambda_i)$, the shares allocate proportional to $e_{T,i}w_i$.
The resulting shares (order $A,C,E$) are:
\begin{itemize}
    \item \textbf{Linear} $g(\lambda)=\lambda$: $\approx (0.164, 0.589, 0.246)$.
    \item \textbf{Power} $g(\lambda)=\lambda^2$: $\approx (0.155, 0.498, 0.348)$.
    \item \textbf{CVaR} $g(\lambda)=(\lambda-0.9)_+$: $\approx (0.149, 0.114, 0.737)$.
\end{itemize}
RAP shifts haircut mass toward high-leverage winners; the tilt is mild for linear $g$, stronger for $\lambda^2$, and concentrates almost entirely on the over-threshold tail for the CVaR model.

\iparagraph{Next deficit and leverage sensitivity.}
Consider the setup where $T=2$ with $D_T \approx 0.705$ and $W_T \approx 7.72$.
Under the normal pro-rata rule, the haircut rate is $h^{\mathrm{PR}}_T \approx 0.0913$.
We consider a simple Markovian shock whose direction is uniformly random and whose magnitude grows with the winner leverage mass:
\[
Z_{T,i} = \xi_T \zeta_T, \quad \xi_T \in \{-1,+1\} \text{ equiprobable}, \quad \zeta_T = \alpha \frac{\ell^+_T}{k_T}.
\]
With $\alpha=1.2$, the expected next deficit is:
\[
\Expect[D^{\mathrm{next}}_{T+1} \mid \mathcal F_T] \approx 1.46 > D_T.
\]
This illustrates a failure mode for pro-rata when the shock kernel scales strongly with leverage: pro-rata shrinks all winners uniformly and leaves effective leverages $\lambda_{T,i}$ unchanged, so the shock magnitude $\zeta_T$ is unaffected while residual exposure remains large on high-leverage winners.

\iparagraph{Correlated shocks example.}
Consider two winners with equal equity $e$ and leverage levels $\lambda_{t,1}>\lambda_{t,2}\ge 1$, and budget $b_t=2\varepsilon e$.
Assume price shocks are AR(1): $Z_{t+2}=\rho Z_{t+1}+\varepsilon_{t+2}$, $\rho\in(0,1)$.
RAP with $w_{t,i} \propto \lambda_{t,i}\psi(1/\lambda_{t,i})$ puts more haircut mass on account 1 (the higher leverage account).
Assume account 1's exposure to $Z_{t+1}$ offsets the loss term (a "hedge") in the next step deficit: $D^{\text{next}}_{t+1}=(\alpha-\beta s_{t,1})Z_{t+1}$.
Shrinking $s_{t,1}$ (winner 1's residual equity) weakens the hedge into $t+2$.
Since $s^{\mathrm{RAP}}_{t,1} < s^{\mathrm{PR}}_{t,1}$, the two-step sum of deficits $S_t$ can satisfy $S_t(h^{\mathrm{PR}}_t) < S_t(h^{\mathrm{RAP}}_t)$, even if RAP minimizes the one-step deficit.

\iparagraph{Exchange incentive compatibility example.}
Consider two positions $\mathfrak{p}_A$ (high value, $\theta_A=100$) and $\mathfrak{p}_B$ (low value, $\theta_B=5$) with equal initial leverage $\ell=1$.
Suppose the exchange must reduce total leverage by 1 unit.
\begin{itemize}
    \item \textbf{RAP (Targeted):} Fully liquidates $\mathfrak{p}_A$. Continuation value: $5\beta$.
    \item \textbf{Pro-Rata:} Reduces both by 50\%. Continuation value: $52.5\beta$.
\end{itemize}
Pro-rata yields significantly higher utility by preserving the high-value trader, demonstrating that RAP need not be incentive compatible for the exchange.


\subsection{RAP Optimality and Convex Dominance}\label{app:rap-optimality-and-convex-dominance}\label{sec:rap-optimality-and-convex-dominance}
In this appendix, we prove two statements: 1) RAP optimizes the one-step next deficit~\eqref{eq:delta-T} and 2) RAP has a smaller residual than any other comonotone haircut rule.

\iparagraph{RAP optimizes $\delta_T$.}
In this section, we briefly show that RAP optimizes the one-step deficit proxy $\delta_T$.
We do this by showing that the weights determined by the perspective transformr $\rho(\lambda)$, which define $g^{\star}$, optimize $\delta_T$.

\begin{proposition}\label{thm:rap-onestep-7}
Fix a round $t$ with budget $b_t=\theta_t|D_t|$ and per‑account caps $0\le H_{t,i}\le 1$. For
\[
\delta_t(h)=\sum_{i\in W_t} (1-h_{t,i})\,\lambda_{t,i}e_{t,i}\,\psi_i\!\Big(\tfrac{1}{\lambda_{t,i}}\Big)
\]
the capped reverse-\-waterfilling with risk weights $\tilde{w}_{t,i}=\rho(\lambda_{t,i})=\lambda_{t,i}\psi_i(1/\lambda_{t,i})$ minimizes $\delta_t(h)$ among all $h$ with $\sum_i w_{t,i}h_{t,i}=b_t$ and $0\le h_{t,i}\le H_{t,i}$, where $w_{t,i}$ is the haircutable endowment (under PNL-only, $w_{t,i}=(\mathrm{PNL}_{t,i})_+$).
\end{proposition}
\begin{proof}
Using $\rho(\lambda)=\lambda\psi(1/\lambda)$,
\[
\delta_t(h)=\sum_i (1-h_{t,i})\,\lambda_{t,i}e_{t,i}\,\psi(1/\lambda_{t,i})
\equiv C_h - \sum_i \rho(\lambda_{t,i}) e_{t,i} h_{t,i}
\]
where $C_h$ is a constant independent of $h$ (can depend on $\lambda_{t,i}$).
Note that $e_{t,i} = c_{t,i} + \mathrm{PNL}_{t,i}$ and under PNL-only haircuts, $w_{t,i} = (\mathrm{PNL}_{t,i})_+$.
The gradient $-\partial \delta_t / \partial x_{t,i}$ where $x_{t,i} = h_{t,i} w_{t,i}$ is $\rho(\lambda_{t,i}) e_{t,i}$ (expressed in equity terms because insolvency is an equity concept).
Maximizing $\sum_i \rho(\lambda_{t,i})\,e_{t,i} (x_{t,i}/w_{t,i})$ under $\sum_i x_{t,i}=b_t$ and $0\le x_{t,i}\le w_{t,i}H_{t,i}$ is a fractional knapsack problem solved by sorting the values $\rho(\lambda_{t,i}) e_{t,i}/w_{t,i}$.
The optimizer for this problem is reverse‑waterfilling~\cite{BoydVandenberghe2004}:
\[
h_{t,i}=\min\{H_{t,i},\ \tau^\star_t\,\tilde{w}_{t,i}\},\qquad \tilde{w}_{t,i}=\rho(\lambda_{t,i}),
\]
with $\tau^\star_t$ set by $\sum_i w_{t,i}h_{t,i}=b_t$. 
This choice of $\tilde{w}$ minimizes $\delta_t$ among all weighted reverse‑waterfilling rules.
\end{proof}

\iparagraph{RAP realizes Schur-convex dominance.}\label{app:lpr-convex-dominance}
We first note that theoretical results from the measure theoretic literature imply than RAP should provide Schur-convex dominance.
RAP weights $w_i = \lambda_{T,i} g(\lambda_{T,i})$ can be interpreted as allocating budget proportional to a coherent, law-invariant risk measure.
Specifically, let $\rho(\lambda) = \lambda\,\psi(1/\lambda)$ be a risk density with $\psi$ convex and nonincreasing.
This corresponds to a spectral risk measure $\rho(X) = \int_0^1 \mathrm{ES}_\alpha(X)\,d\mu(\alpha)$ via the Kusuoka representation~\citep{Kusuoka2001,Acerbi2002}.
Choosing $g(\lambda) = \rho(\lambda)/\lambda$ aligns the RAP allocation with this spectral density.
Known results from~\cite{Kusuoka2001} then imply RAP with a ``more convex'' risk density (in the Schur sense) will weakly Schur-dominate any other weighted pro-rata rule with a less concentrated density.
Instead of utilizing such strong measure theoretic tools, we instead prove this directly below in an elementary manner.

\begin{theorem}[Constructive Schur–convex dominance]\label{thm:rho-to-g-dominance}
Fix $T$, budget $b_T$, and caps $(\beta_i)$. Let $\rho$ be nondecreasing and $g^\star(\lambda)=\rho(\lambda)/\lambda$.
Consider any weighted pro-rata rule $h^{(w)}$ with weights $w$.
If the haircut share vector of $h^{(w)}$ is no more concentrated on high-$\rho$ indices than that of $\mathrm{RAP}(g^\star)$ on any fixed active set, then the residual vector of $\mathrm{RAP}(g^\star)$ weakly submajorizes that of $h^{(w)}$:
\[
z_T(\mathrm{RAP}(g^\star)) \preceq_w z_T(h^{(w)}).
\]
Thus, for any convex increasing $\phi$, $\sum \phi(z_{T,i}(\mathrm{RAP})) \le \sum \phi(z_{T,i}(h^{(w)}))$.
\end{theorem}

\begin{proof}
We split this proof into three steps.
The first step is analogous to the proof of Proposition~\ref{thm:unique-pro-rata}, where we analyze the change to the weights on a piecewise constant set of intervals.
The second step is to analyze how the residuals change with budget using the piecewise-constant representation.
Given the change in residual, the final step uses the majorization inequality to show that the residual vector of RAP weakly submajorizes that of any other weighted pro-rata rule.

\emph{Step 1: Active set and parameterization.}
We first define the active set $A(b)$ as in~\eqref{eq:active-set}.
Any weighted reverse‑waterfilling with risk weights $\tilde{w}$ admits the water‑level form
\[
  h^{(\tilde{w})}_{T,i}(b)\;=\;\min\{\beta_i,\ \tau^{(\tilde{w})}(b)\,\tilde{w}_i\},
\]
where $\tau^{(\tilde{w})}(b)$ is chosen so that $\sum_i w_{T,i}\,h^{(\tilde{w})}_{T,i}(b)=b$ (budget constraint in endowment space).
On an interval $[b_0,b_1]$ with fixed $A(b)$ we have
\[
  \frac{db}{d\tau^{(\tilde{w})}}=\sum_{j\in A(b)} w_{T,j} \tilde{w}_j,
  \qquad
  \frac{d h^{(\tilde{w})}_{T,i}}{db}
  \;=\;
  \begin{cases}
    \displaystyle \frac{\tilde{w}_i}{\sum_{j\in A(b)} w_{T,j} \tilde{w}_j},& i\in A(b),\\[1.0ex]
    0,& i\notin A(b).
  \end{cases}
\]
\emph{Step 2: Residual dynamics.}
Next we look at how the residuals (\eg~post haircut equity) change with budget.
Write the reweighted residuals as
\[
  z_{T,i}(b)\;=\;\rho(\lambda_{T,i})\,e_{T,i}\,\bigl(1-h_{T,i}(b)\bigr),
\]
where the equity $e_{T,i}$ appears because insolvency is an equity concept, but the control variable is the endowment $w_{T,i}$ via $h_{T,i}$.
Then on $[b_0,b_1]$,
\[
  \frac{d z^{(\tilde{w})}_{T,i}}{db}
  \;=\;
  \begin{cases}
    \displaystyle -\frac{\rho(\lambda_{T,i})\,e_{T,i}\,\tilde{w}_i}{\sum_{j\in A(b)} w_{T,j} \tilde{w}_j},& i\in A(b),\\[1.0ex]
    0,& i\notin A(b).
  \end{cases}
\]
For $\mathrm{RAP}(g^\star)$ we take $\tilde{w}_i=\rho(\lambda_{T,i})$, giving
\[
  \frac{d z^{(\mathrm{RAP})}_{T,i}}{db}
  \;=\;
  \begin{cases}
    \displaystyle -\frac{\rho(\lambda_{T,i})^2\,e_{T,i}}{\sum_{j\in A(b)} w_{T,j} \rho(\lambda_{T,j})},& i\in A(b),\\[1.0ex]
    0,& i\notin A(b).
  \end{cases}
\]
\emph{Step 3: Majorization at each budget.} On a fixed $A(b)$, sort indices by decreasing $\rho(\lambda_{T,i})$.
By the hypothesis that the haircut share vector of $h^{(\tilde{w})}$ on $A(b)$,
\[
  \sigma^{(\tilde{w})}_i(b)\ :=\ \frac{w_{T,i} \tilde{w}_i}{\sum_{j\in A(b)} w_{T,j} \tilde{w}_j},
\]
is no more concentrated on high‑$\rho$ indices than the RAP share
$\sigma^{(\mathrm{RAP})}_i(b)=\frac{w_{T,i} \rho(\lambda_{T,i})}{\sum_{j\in A(b)} w_{T,j} \rho(\lambda_{T,j})}$, the rearrangement/majorization inequality implies that for every $k$,
\[
  \sum_{i\le k}\!\frac{d z^{(\mathrm{RAP})}_{T,(i)}}{db}
  \;\leq\;
  \sum_{i\le k}\!\frac{d z^{(\tilde{w})}_{T,(i)}}{db}
\]
where $(i)$ denotes the order by decreasing $\rho$.
Hence the instantaneous decrease vector under RAP weakly submajorizes that under $h^{(\tilde{w})}$ on $[b_0,b_1]$.
Integrating over $b$ preserves $\prec_w$ on the interval, and concatenating the finitely many intervals where $A(b)$ changes preserves the order overall:
$z_T(\mathrm{RAP}(g^\star)) \preceq_w z_T(h^{(\tilde{w})})$.
Schur–convexity then yields the separable convex loss comparison.
\end{proof}


\subsection{Example of the Solvency vs. Long-Term Revenue Trade-Off}\label{app:exchange-incentives}
This example illustrates a fundamental tension: the risk-minimizing policy (RAP) may be suboptimal for the exchange's long-term value (LTV) because it disproportionately liquidates high-leverage users who generate the most fees. Under certain conditions, a ``fairer'' policy like Pro-Rata (PR), which preserves these high-value users, yields higher total utility for the exchange.

\iparagraph{Setup.}
Consider an exchange with two profitable users $i\in\{H,L\}$. User $H$ is high-leverage ($\lambda_H > \lambda_L$) and high-revenue; user $L$ is safer but generates less fee volume.
The exchange must raise a budget $b$ via haircuts $h=(h_H, h_L)$ to cover a deficit.
Its objective combines immediate safety (minimizing insolvency risk) and future revenue (LTV):
\[
U^{\mathrm{exch}}(h)\;=\;\underbrace{-\mathrm{Loss}(h)}_{\text{Immediate Safety}}\;+\;\underbrace{\beta\sum_{i\in\{H,L\}} \theta_i(1-h_i)\lambda_i}_{\text{Future Revenue (LTV)}},
\]
where $\mathrm{Loss}(h) = L_0 - \alpha_H h_H - \alpha_L h_L$ is the expected insurance fund draw, and $\theta_i \lambda_i$ is the expected future fee revenue per unit of equity from user $i$.
We assume $\alpha_H/e_H > \alpha_L/e_L$, meaning user $H$ provides the cheapest risk reduction per dollar of haircut.

\iparagraph{Policy comparison.}
We compare two policies:
\begin{itemize}
    \item \emph{RAP (Risk-Minimizing):} Prioritizes risk reduction above all.
    Since $H$ offers the best ``bang for the buck'' in safety ($\alpha_H/e_H > \alpha_L/e_L$),\footnote{The marginal reduction in loss per unit of budget is $\frac{\partial \mathrm{Loss}}{\partial h_i} \frac{d h_i}{d (\text{budget})} = \frac{\alpha_i}{e_i}$. Since user $H$ has higher leverage, they have a higher risk coefficient $\alpha_H$, making $\alpha_H/e_H$ the steepest descent direction for the loss function.} RAP fully targets $H$ first: $h^{\mathrm{RAP}} = (b/e_H, 0)$ (assuming $b < e_H$).
    \item \emph{Pro-Rata (Revenue-Preserving):} Spreads the pain evenly, setting $h^{\mathrm{PR}}_i = \frac{b}{e_H+e_L}$ for both users. This is less efficient for immediate safety but preserves more of user $H$'s position.
\end{itemize}

\iparagraph{When an exchange prefers Pro-Rata to maximize long-term revenue.}
The exchange prefers PR over RAP when the LTV gain from saving user $H$ outweighs the increased immediate risk.
The utility difference is:
\[
\Delta U = U^{\mathrm{exch}}(h^{\mathrm{PR}}) - U^{\mathrm{exch}}(h^{\mathrm{RAP}}) 
\;=\; \underbrace{\Delta h_H (\beta \theta_H \lambda_H - \alpha_H)}_{\text{Gain from saving } H} \;-\; \underbrace{h^{\mathrm{PR}}_L (\alpha_L - \beta \theta_L \lambda_L)}_{\text{Cost of cutting } L}.
\]
If user $H$ is sufficiently profitable ($\theta_H$ is large), then $\Delta U > 0$.
Specifically, PR dominates RAP if the relative revenue of the high-leverage user exceeds a threshold:
\[
\frac{\theta_H}{\theta_L} \;\ge\; \Theta^\star\;=\;\frac{h^{\mathrm{PR}}_L}{\Delta h_H}\cdot\frac{\lambda_L}{\lambda_H}\;+\;\frac{\alpha_H-\frac{h^{\mathrm{PR}}_L}{\Delta h_H}\alpha_L}{\beta\,\lambda_H\,\theta_L}.
\]
While RAP is ``optimal'' for preventing immediate insolvency, it can be myopic. If high-leverage traders are the exchange's cash cows, the exchange has a rational incentive to use Pro-Rata to keep them active, even at the cost of slightly higher short-term risk.

\end{document}
