\section{Liquidations, Autodeleveraging, and Insurance Funds}\label{app:liquidation-mechanics}

\subsection{Bankruptcy Price Example}
We illustrate the bankruptcy price calculation with an example.
Fix $\ell^{\max}=10$ (so $m_I=0.10$) and $p_0=1$.
Consider the five running positions from $\mathcal{P}_5$:
\begin{align*}
\mathfrak p_A&=(q,c,b)=(1,\,2,\,+1), &
\mathfrak p_B&=(1,\,2/3,\,+1), &
\mathfrak p_C&=(4,\,8/3,\,-1),\\
\mathfrak p_D&=(1,\,2/19,\,+1), &
\mathfrak p_E&=(1,\,10/99,\,-1).
\end{align*}
Applying Eq.~\eqref{eq:bankruptcy-price-no-funding} with $\Gamma=0$ and $p_t=p_0=1$ gives
\begin{align*}
 p^{bk}(\mathfrak p_A)&=\max\{0,\,1-2\}=0,\\
 p^{bk}(\mathfrak p_B)&=1-\tfrac{2}{3}=\tfrac{1}{3},\\
 p^{bk}(\mathfrak p_C)&=1-\tfrac{\,8/3\,}{-4}=1+\tfrac{2}{3}=\tfrac{5}{3}\ (1.6667),\\
 p^{bk}(\mathfrak p_D)&=1-\tfrac{2}{19}\approx 0.8947,\\
 p^{bk}(\mathfrak p_E)&=1+\tfrac{10}{99}\approx 1.1010.
\end{align*}
Thus A is robust to a full drop to zero; B (long 1.5x) has a low bankruptcy price; C (short 1.5x) bankrupts only if the mark rises above $\tfrac{5}{3}$; D/E (long ~9.5x/9.9x) have high bankruptcy prices close to $1$, making negative equity likely if liquidations lag.

\subsection{Liquidation Price Example}
For $\mu=0.10$ the liquidation prices evaluate to factors $\tfrac{1}{1-\mu}$ for longs and $\tfrac{1}{1+\mu}$ for shorts. Using the same five positions with $p_{t_i}=1$ and $\Gamma=0$,
\begin{align*}
 \hat p^{liq}(\mathfrak p_A,0.10)&=\tfrac{1}{0.9}\cdot 0=0,\\
 \hat p^{liq}(\mathfrak p_B,0.10)&=\tfrac{1}{0.9}\cdot \tfrac{1}{3}\approx 0.3704,\\
 \hat p^{liq}(\mathfrak p_C,0.10)&=\tfrac{1}{1.1}\cdot \tfrac{5}{3}\approx 1.5152,\\
 \hat p^{liq}(\mathfrak p_D,0.10)&=\tfrac{1}{0.9}\cdot \tfrac{17}{19}\approx 0.9942,\\
 \hat p^{liq}(\mathfrak p_E,0.10)&=\tfrac{1}{1.1}\cdot \tfrac{109}{99}\approx 1.0010.
\end{align*}
Long A would liquidate only if the oracle fell to $0$ (\ie~never occurs in practice). The short $C$ becomes liquidatable only when the oracle exceeds its threshold; the high‑leverage long $D$ and short $E$ become liquidatable close to $1$ if collateral is not topped up.

\subsection{Execution Price Example}
We summarize execution with a directional linear impact rule consistent with our notation: selling to close a long uses $P^{sell}(x)=p_t-\alpha x$ and buying to close a short uses $P^{buy}(x)=p_t+\alpha x$ with $\alpha>0$; the volume–weighted execution for a slice is
$\;p^{exec}=p_t-\tfrac{\alpha}{2}\Delta q$ if $b_i=+1$ (sell) and $p^{exec}=p_t+\tfrac{\alpha}{2}\Delta q$ if $b_i=-1$ (buy).
Directional linear impact with a single $\alpha$: selling (closing a long) uses $P^{sell}(x)=p_t-\alpha x$, buying (closing a short) uses $P^{buy}(x)=p_t+\alpha x$. The slice VWAP over $[0,\Delta q]$ is $p^{exec}=p_t\mp\tfrac{\alpha}{2}\Delta q$ (minus for sells, plus for buys). Fix $\alpha=1.0$ and choose $\Delta q$ per case:
\begin{itemize}[leftmargin=12pt]
  \item $\mathfrak p_A$ (Long): $p_t=1.30$, $\Delta q=0.5$ gives $p^{exec}=1.30-0.5\cdot 0.5=1.05$. Here $p^{bk}(\mathfrak p_A)=\max\{p_t-2,0\}=0$, so $p^{exec}>p^{bk}$.
  \item $\mathfrak p_B$ (Long): $p_t=0.95$, $\Delta q=0.2$ gives $p^{exec}=0.95-0.5\cdot 0.2=0.85$. With $p^{bk}(\mathfrak p_B)=\max\{0.95-\tfrac{2}{3},0\}\approx 0.2833$, we have $p^{exec}>p^{bk}$.
  \item $\mathfrak p_C$ (Short): $p_t=1.60$, $\Delta q=2.0$ gives $p^{exec}=1.60+0.5\cdot 2.0=2.60$. Since $p^{bk}(\mathfrak p_C)=p_t+\tfrac{8/3}{4}=p_t+\tfrac{2}{3}=2.2667$, $p^{exec}>p^{bk}$ (adverse for a short).
  \item $\mathfrak p_D$ (Long; targeted): $p_t=0.98$, $\Delta q=0.4$ gives $p^{exec}=0.98-0.5\cdot 0.4=0.78$. With $p^{bk}(\mathfrak p_D)=p_t-\tfrac{2}{19}\approx 0.8747$, we achieve $p^{exec}<p^{bk}$.
  \item $\mathfrak p_E$ (Short): $p_t=1.05$, $\Delta q=0.4$ gives $p^{exec}=1.05+0.5\cdot 0.4=1.25$. With $p^{bk}(\mathfrak p_E)=p_t+\tfrac{10}{99}\approx 1.151$, we have $p^{exec}>p^{bk}$.
\end{itemize}

\subsection{Liquidation Costs Example}
To ground the fee model, let $\tau(\Delta q)=\tau^{fix}+\phi^{mark} p_t\,\Delta q+\phi^{exec} p^{exec}\,\Delta q$ as in practice.
Consider $p_t=1.30$, a slice $\Delta q=0.50$, and a realized $p^{exec}=1.32$.
Two parameterizations:
\begin{itemize}[leftmargin=12pt]
  \item Binance: $(\tau^{fix},\phi^{mark},\phi^{exec})=(0,\,40\,\mathrm{bps},\,0)$ \citep{BinanceFuturesInsuranceFund}. Then $\tau=0.0040\cdot 1.30\cdot 0.50=0.0026$.
  \item Hyperliquid: $(\tau^{fix},\phi^{mark},\phi^{exec})=(0,\,20\,\mathrm{bps},\,10\,\mathrm{bps})$ \citep{HyperliquidDocsLiquidations}. Then $\tau=0.0020\cdot 1.30\cdot 0.50+0.0010\cdot1.32\cdot0.50\approx0.00130+0.00066=0.00196$.
\end{itemize}
\noindent Rates and formulas vary by venue and contract; the above are illustrative parameterizations consistent with public documentation that liquidation fees are charged and, on centralized venues like Binance, credited to the insurance fund.

\subsection{Liquidation Strategy Example}
Consider short $\mathfrak p_E$ when the mark jumps to $p_t=5.5$ (ignore funding for this step).
Equity before liquidation is $e\approx c_E - q_E(p_t-p_0)=\tfrac{10}{99}-1\cdot 4.5\approx -4.399$.
Let $\mu=0.10$ and a linear fee $\tau(\Delta q)=\phi\,p_t\,\Delta q$ with $\phi=30$bps. Suppose execution is $p^{exec}=5.55$.
Using \eqref{eq:liquidation-strategy} and $b=-1$, the minimal slice that restores maintenance solves
\[
\Delta q\;=\; \frac{\mu p_t q - e}{\,b(p^{exec}-p_t) - \phi p_t + \mu p_t\,}
\;=\; \frac{0.1\cdot 5.5\cdot 1 - (-4.399)}{-0.05 - 0.003\cdot 5.5 + 0.1\cdot 5.5}\ \approx\ \frac{4.949}{0.4835}\ \approx\ 10.24.
\]
Since $\Delta q>q_E$, a greedy policy would fully close E (cap at $\Delta q=q_E=1$).

\subsection{Bad Debt Example}
Consider the high‑leverage long $\mathfrak p_D$ and a slice of size $\Delta q=0.4$ at $p_t=0.98$.
Suppose the realized execution is $p^{exec}_D=0.78$ while the bankruptcy level is $p^{bk}_D\approx 0.8747$.
Since $p^{exec}_D<p^{bk}_D$, the realized shortfall from this slice is
\[
 (p^{bk}_D-p^{exec}_D)\,\Delta q\ \approx\ (0.8747-0.78)\cdot 0.4\ \approx\ 0.0379,
\]
which contributes this amount to the period bad debt $D_t$ (cf. Eq.~\eqref{eq:total-bad-debt}).
Coverage follows the solvency waterfall: the insurance fund pays $\min\{\mathsf{IF}_t, D_t\}$ and any residual shortfall is socialized via ADL (see \S\ref{subsec:exchange-solvency}, \S\ref{subsec:adl}).
The short case is symmetric: buying to close above bankruptcy ($p^{exec}>p^{bk}$) realizes a positive contribution to $D_t$.

\subsection{Anatomy of a Liquidation}
Given the bankruptcy, liquidation, and execution prices, we can now describe the high-level algorithm that liquidations follow.
We note that many live liquidation systems will have much more complex liquidation algorithms.
These complexities deal with the coordination costs of coordinating many parties (\eg~oracle provider, liquidators, spot order book liquidity) and precise models that exchanges use for their liquidation strategy.
However, we effectively lump all of these complexities into the definition of the liquidation strategy.
The following liquidation loop is run on every oracle update received by a perpetuals exchange: 
\begin{itemize}
\item For $\mathfrak{p}_{i,t} \in \mathcal{P}_n$
\begin{itemize}
\item If the maintenance margin condition~\eqref{eq:maintenance-margin} is violated for $\mathfrak{p}_{i,t}$
\begin{enumerate}
    \item Remove the position $\mathcal{P}_n \leftarrow \mathcal{P}_n - \{\mathfrak{p}_{i, t}\}$
    \item Estimate quantity to liquidate $\Delta q_i \leftarrow L(\mathfrak{p}_{i,t}, p_{1:T}, \hat{p}_{1:T})$
    \item Liquidator executes $\Delta q_i$-sized liquidation and returns their execution price $p^{exec}(\Delta q_i)$
    \item Update position: $\mathfrak{p}'_{i,t} = (q_i - \Delta q_i, c_i + p^{exec} \Delta q_i - \tau_t(\Delta q_i), t_i, b_i)$
    \item Re-add the position position: $\mathcal{P}_n \leftarrow \mathcal{P}_n \cup \{\mathfrak{p}'_{i, t}\}$
    \item Update equity using~\eqref{eq:adjusted-equity}
\end{enumerate}
\end{itemize}
\item If $\mathfrak{p}_{i,t}$ has bad debt, $\tilde{e}(\mathfrak{p}_{i,t}, p_{1:T}, \hat{p}_{1:T}, \Delta q_i) < 0$, then
\begin{itemize}
    \item Attempt to use the insurance fund, if it exists, to cover the bad debt (\S\ref{subsec:exchange-solvency})
    \item If the insurance fund is insufficiently sized, utilize an ADL mechanism (\S\ref{subsec:adl})
\end{itemize}
\end{itemize}

\iparagraph{Example.}
We illustrate a five–step path using the running set $\mathcal{P}_5$ from above. Take $T=5$, $p_0=1$ and
\[
 p_{0:5}=(1.00,\ 0.96,\ 0.94,\ 0.97,\ 1.05,\ 1.12),\qquad \hat p_t=p_t\ \ (t=0,\dots,5),\qquad \mu=m_I=0.10.
\]
Executions follow the directional linear impact rule introduced in the execution example: for a slice of size $\Delta q$ at time $t$, the volume‑weighted execution is
\[
\begin{aligned}
  p^{exec}\;=\;p_t-\tfrac{\alpha}{2}\,\Delta q&\quad\text{(sell to close a long, $b=+1$)},\\
  p^{exec}\;=\;p_t+\tfrac{\alpha}{2}\,\Delta q&\quad\text{(buy to close a short, $b=-1$)}.
\end{aligned}
\]
with $\alpha>0$. We take $\alpha=1.0$ and choose $\Delta q$ via the loop's liquidation size $\Delta q_i=L(\mathfrak{p}_{i,t}, p_{1:T}, \hat p_{1:T})$.

\iparagraph{D liquidates at $t=2$ (no bad debt).}
At $t=2$ we have $p_2=0.94$ and the maintenance condition \eqref{eq:maintenance-margin} is violated for $\mathfrak p_{D,2}$, so the loop attempts a partial liquidation. Take $\Delta q_D=L(\mathfrak{p}_{D,2},\cdot)=0.20$ for illustration. By \eqref{eq:bankruptcy-price-no-funding},
\[
 p^{bk}(\mathfrak p_{D,2})\;=\;p_2-\tfrac{2}{19}\;\approx\;0.8347,\qquad
 p^{exec}_D\;=\;p_2-\tfrac{\alpha}{2}\Delta q_D\;=\;0.94-0.10\;=\;0.84.
\]
Since $p^{exec}_D>p^{bk}(\mathfrak p_{D,2})$, this slice executes without bad debt; the position is updated to $\mathfrak{p}'_{D,2}=(q_D-\Delta q_D,\ c_D+p^{exec}_D\Delta q_D-\tau_2(\Delta q_D),\ t_D,\ b_D)$ and equity is updated per \eqref{eq:adjusted-equity} before reinserting $\mathfrak{p}'_{D,2}$ into $\mathcal{P}_n$.

\iparagraph{E becomes bad debt at $t=4$ (short; liquidation fails).}
At $t=4$ we have $p_4=1.05$ and \eqref{eq:maintenance-margin} is violated for $\mathfrak p_{E,4}$ with $b_E=-1$. The loop selects a liquidation size; take a full close $\Delta q_E=L(\mathfrak{p}_{E,4},\cdot)=1$. By \eqref{eq:bankruptcy-price-no-funding},
\[
 p^{bk}(\mathfrak p_{E,4})\;=\;p_4+\tfrac{10}{99}\;\approx\;1.1510,\qquad
 p^{exec}_E\;=\;p_4+\tfrac{\alpha}{2}\Delta q_E\;=\;1.05+0.50\;=\;1.55.
\]
For a short, $p^{exec}>p^{bk}$ realizes bad debt. The loop records the shortfall
\[
 D_4\;=\;\big(p^{exec}_E-p^{bk}(\mathfrak p_{E,4})\big)\,\Delta q_E\;\approx\;0.399,\qquad \tilde e\big(\mathfrak p_{E,4}, p_{1:5}, \hat p_{1:5}, \Delta q_E\big)\;=\;-D_4<0,
\]
and then attempts coverage via the insurance fund (up to $\min\{\mathsf{IF}_4,D_4\}$); any residual shortfall $R_4$ defined by Eq.~\eqref{eq:adl-residual} is socialized by ADL (see \S\ref{subsec:adl}).


\subsection{Optimal Capital Structure Derivation}\label{app:optimal-capital}
In this section, we compute the optimal static insurance fund size $IF^\star$ that trades off the opportunity cost of capital and expected uncovered losses beyond $IF$.

\paragraph{Setup.} Let $D_T$ denote the round deficit with pdf $f_D$ and tail $\bar F_D(x)=\Pr[D_T>x]$. Let $r>0$ be the per‑unit capital cost and $\kappa>0$ the per‑unit social loss weight of uncovered deficits.
The objective is
\[
\min_{IF \ge 0}\ \mathcal{J}(IF)\;=\; r\,IF\;+\;\kappa\,\mathbb{E}\big[(D_T-IF)_+\big]
\]
which equals $r\,IF+\kappa\int_{IF}^{\infty}(x-IF)f_D(x)\,dx$ when $D_T$ is continuous.

\paragraph{Optimality condition.} Differentiating yields,
\[
\mathcal{J}'(IF)= r - \kappa\,\bar F_D(IF).
\]
Hence any interior minimizer satisfies $\bar F_D(IF^\star)=r/\kappa$, i.e.,
\[
IF^\star \;=\; \bar F_D^{-1}\!\Big(\tfrac{r}{\kappa}\Big)\;=\;\mathrm{VaR}_{\,1-r/\kappa}(D_T).
\]

\paragraph{Assumptions.}
In order for this argument to hold, we assume that $\mathcal J$ is convex and differentiable.
Moreover, if $r\ge \kappa$ then we define $IF^\star=0$, whereas if $r\to 0$, then $IF^\star \rightarrow \sup D_T$.